\documentclass[]{marticle}
\usepackage[utf8]{inputenc}
\usepackage[italian]{babel}
\usepackage{amssymb}
\usepackage{mstyle}

\title{\textbf{\Huge Relazione di Laboratorio Computazionale}}
\date{}


\begin{document}
\maketitle

\section*{Abstract}
In questa relazione prenderemo in considerazione il problema di estrarre
campioni di valori casuali data una certa distribuzione di probabilita` discreta.
Supporremo di sapere generare variabili uniformi sull'intervallo $[0,1]$, che
verranno implementate operativamente come le variabili generate dalla libreria
\code{numpy}.

Una prima soluzione del problema \`e quella di dividere $[0,1]$ in intervalli di
lunghezza pari a ciascuna componente del vettore di probabilit\`a, e scegliere
il risultato in funzione dell'intervallo a cui appartiene una variabile
uniforme. Questo presenta diversi inconvenienti: il metodo infatti richiede un
numero di somme proporzionale al numero di componenti del vettore di
probabilit\`a. Questo potrebbe essere intrattabile quando molto grande. Inoltre
spesso il vettore di probabilit\`a \`e noto solo a meno di un coefficiente di
normalizzazione, il cui calcolo richiederebbe di nuovo $O(n)$ somme. 

Uno dei metodi pi\`u utilizzati per ovviare a questi problemi \`e il metodo di
Monte Carlo (abbreviato spesso con MCMC, Markov Chain Monte Carlo), che consiste
nella simulazione di una camminata su una catena di Markov con distribuzione
invariante la distribuzione data. Dopo un numero sufficiente di step, la
frequenza di visita di un nodo sar\`a arbitrariamente vicina a quella voluta. In
questo caso per\`o il numero di passi necessari ad una determinata distribuzione
non \`e noto a priori ed \`e di difficile calcolo.

Si andr\`a dunque a presentare l'algoritmo di Propp-Wilson, una modifica del
MCMC che ha il vantaggio di ottenere la distribuzione esatta e di terminare una
volta raggiunta questa. Applicheremo tale algoritmo al modello di Ising, una
modellizzazione del comportamento magnetico della materia.

\section{Il modello di Ising}
Sia $G=(V,E)$ un grafo. I vertici andranno a rappresentare i singoli atomi di un
materiale, e gli archi indicano quali atomi interagiscono fra loro. Ad ogni
atomo viene quindi associato uno spin che pu\`o essere $+1$ o $-1$. Una
configurazione \`e quindi una funzione $f\colon V \rightarrow \{+1, -1\}$. A
ciascuno di questi modelli si associa l'energia 
\[
    H(f) = \sum_{(x,y)\in E} f(x)f(y)
\].
Inoltre viene dato un parametro reale del sistema $beta \geq 0$ detta
temperatura inversa. Il modello di ising associa ad ogni configurazione $f \in
\{+1, -1\}^V$ la probabilit\`a
\[
    \pi(f) = \frac{1}{Z} \exp(-\beta H(f))
\]
Dove $Z$ \`e il coefficiente di normalizzazione pari a
\[
    Z = \sum_{f \in \{+1,-1\}^V} \exp(-\beta H(f))
\]


\end{document}
